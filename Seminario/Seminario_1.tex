% Compilar con LuaLaTeX o XeLaTeX
\documentclass[10pt]{article}

% ====== Configuración básica ======
\usepackage[spanish]{babel}
\usepackage{geometry}
\geometry{letterpaper, margin=2.5cm}

% Tipografías (Arial o fallback a TeX Gyre Heros)
\usepackage{fontspec}
\IfFontExistsTF{Arial}{
  \setmainfont{Arial}
}{
  \setmainfont{TeX Gyre Heros} % similar a Arial/Helvetica
}

% Paquetes útiles
\usepackage{microtype}   % mejor espaciado tipográfico
\usepackage{graphicx}
\usepackage{booktabs}
\usepackage{float}
\usepackage{hyperref}
\usepackage[dvipsnames]{xcolor}
\usepackage{enumitem}
\usepackage{setspace}
\usepackage{titlesec}
\usepackage{fancyhdr}

% Enlaces discretos
\hypersetup{
  colorlinks=true,
  linkcolor=black,
  urlcolor=MidnightBlue,
  citecolor=black
}

% Párrafo y líneas
\setlength{\parindent}{0pt}
\setlength{\parskip}{6pt}
\linespread{1.0}

% ====== Encabezado y pie minimalistas ======
\pagestyle{fancy}
\fancyhf{}
\lhead{\footnotesize Universidad de Valparaíso}
\rhead{\footnotesize Seminario de Investigación — 2º Semestre 2025}
\cfoot{\footnotesize \thepage}
\renewcommand{\headrulewidth}{0.3pt}
\renewcommand{\footrulewidth}{0pt}

% ====== Títulos con aire y jerarquía ======
\titleformat{\section}
  {\large\bfseries}
  {}{0pt}{}
\titlespacing*{\section}{0pt}{10pt}{4pt}

\titleformat{\subsection}
  {\normalsize\bfseries}
  {}{0pt}{}
\titlespacing*{\subsection}{0pt}{8pt}{3pt}

% ====== Metadatos (para reusar) ======
\newcommand{\titulodeltrabajo}{De Señales a Redes: Entendiendo la conectividad cerebral y su función}
\newcommand{\autor}{Cristian Salgado Torres}
\newcommand{\correo}{cristian.salgado@estudiantes.uv.cl}
\newcommand{\rut}{20.973.317-K}
\newcommand{\curso}{Magíster en Ciencias de la Ingeniería Biomédica}
\newcommand{\profesora}{Débora Buendía Palacios}
\newcommand{\fechaentrega}{28 de agosto de 2025}
\newcommand{\expositor}{Joana Cabral}

\begin{document}

% ====== Portada compacta y limpia ======
\begin{center}
  {\LARGE \textbf{\titulodeltrabajo}}\\[8pt]
  {\large \autor}\\[2pt]
  \correo \quad | \quad \rut\\[6pt]
  \curso\\
  Profesora: \profesora\\
  \fechaentrega
\end{center}

\vspace{0.8em}
\hrule height 0.4pt
\vspace{0.8em}

% ====== Contenido según pauta ======
\section*{Identificación}
\textbf{Expositora:} \expositor \quad | \quad
Biomedical Engineer and Neuroscientist

\section*{Introducción}
La resonancia magnetica funcional

\section*{Desarrollo del tema}
Explique en qué consistió la charla y el (los) experimento(s) realizado(s) por el/la investigador(a). Puede incluir figuras o tablas si corresponde.

\section*{Discusión}
Realice un análisis comparativo entre lo expuesto y el contexto de la literatura/área. Señale fortalezas, limitaciones y posibles mejoras/metodologías alternativas.

\section*{Conclusión}
Comente su apreciación sobre el tema y sobre el/los experimento(s) presentados. Señale el principal aprendizaje y proyección.

\section*{Referencias}
\begin{itemize}[leftmargin=0.7cm,itemsep=2pt]
  \item Apellido, N. (Año). \textit{Título del artículo}. Revista/Editorial. DOI/URL.
  \item Autor, M. (Año). \textit{Recurso adicional}. Editorial.
\end{itemize}


\end{document}